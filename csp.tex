\documentclass{amsart}
\usepackage[parfill]{parskip} 
\usepackage{amsmath,amssymb,amsthm}
\usepackage[pdftex]{hyperref}
\usepackage{euscript}
\usepackage{tikz-cd}
\usepackage{appendix}
\usepackage{comment}


\hypersetup{
  colorlinks,
  citecolor=blue,
  filecolor=black,
  linkcolor=blue,
  urlcolor=black
}

\theoremstyle{plain}
\newtheorem{theorem}{Theorem}[section]
\newtheorem{corollary}[theorem]{Corollary}
\newtheorem{lemma}[theorem]{Lemma}


\theoremstyle{definition}
\newtheorem{definition}[theorem]{Definition}

\theoremstyle{remark}
\newtheorem{remark}[theorem]{Remark}
\newtheorem{example}[theorem]{Example}



\def\phi{\varphi}
\def\E{\exists}
\def\A{\forall}
\DeclareMathOperator{\Clo}{Clo}
\DeclareMathOperator{\Con}{Con}
\DeclareMathOperator{\CSP}{CSP}
\DeclareMathOperator{\Inv}{Inv}
\DeclareMathOperator{\Pol}{Pol}
\DeclareMathOperator{\NP}{NP}
\DeclareMathOperator{\id}{id}
\DeclareMathOperator{\typ}{typ}
\DeclareMathOperator{\End}{End}
\DeclareMathOperator{\Sym}{Sym}
\DeclareMathOperator{\Alg}{Alg}
\DeclareMathOperator{\Eq}{Eq}
\DeclareMathOperator{\fin}{fin}

\begin{document}
\title{CSP Fast Track}
\author{Arturo}


\begin{abstract}
   In this note 
\end{abstract}

\maketitle

\section{Introduction}
Let $R$ be a set of relation symbols. 
Let $\EuScript{A}
%=(A, P)
$ be a relational structure over $R$.
Let $X$ be a countable set of variables. 
By the \textbf{constraint satisfaction problem} $\CSP(\EuScript{A})$\footnote{More often denoted by $\CSP(P)$.} we mean the following decision problem: 
given a finite set $\Sigma$ of atomic formulas over $R$, decide whether there is 
%find (or state that it is impossible to find) 
an assignment $(-)^{\EuScript{A}} : X \to A$ such that 
$\EuScript{A} \models \Sigma$; 
i.e. for all $r \in R_n$ and for all $x,y,x_1, \ldots, x_n \in X$ 
\begin{align}
    r(x_1, \ldots, x_n) \in \Sigma & \implies  (x_1^{{\EuScript{A}}}, \ldots, x_n^{{\EuScript{A}}}) \in r^{\EuScript{A}} \\
    x \equiv y \in \Sigma & \implies x^{{\EuScript{A}}} = y^{{\EuScript{A}}}
\end{align}
Clearly, it is enough to find an assignment only for those variables that appear in $\Sigma$. 

%Important: if a set of relation symbols is not fixed, a relational structure $\EuScript{A}$ will be always thought as a structure over the set of its relations. 
\begin{remark}
    Usually we deal with \emph{indexed} relational structures, 
    that is we fix a set of relation symbols $R$ and we consider a set $A$ with a set $P$ of relations on $A$ indexed by the elements of $R$. 
    Sometimes is useful to deal with \emph{non-indexed} relational structures $(A,P)$. 
    In this case $P$ will serve as the index set as well. 
    The same applies to algebraic structures (i.e. algebras) and function symbols.  
\end{remark}
%Starting point: consider the case when $A$ is finite. 
%{The set of relations is finite?}

%We shall say that $\CSP({\EuScript{A}})$ is decidable if there is an algorithm deciding $\CSP({\EuScript{A}})$ for every $\Sigma$ over $R$. 

\begin{definition}
    Let $\rho \in A^k$, and $\phi: A^n \to A$. 
    We say that $\phi$ \textbf{preserves} $\rho$ if given a matrix 
    \begin{equation*}
        \begin{bmatrix}
          a_1^1 & \cdots & a^1_n \\
          \vdots & \ddots &  \vdots  \\
          a^k_1 & \cdots & a^k_n
          \end{bmatrix}
      \end{equation*}
      with $(a_1^1, \ldots, a^k_1) \in \rho, \ldots, (a_1^n, \ldots, a^k_n) \in \rho$, 
      \begin{equation*}
        (\phi(a_1^1 , \ldots , a^1_n), \ldots, \phi(a^k_1 , \ldots , a^k_n)) \in \rho
      \end{equation*}
\end{definition}

\begin{definition}
    If $\Gamma$ is a set of relations on $A$ and $\Phi$ is a set of operations on $A$ we denote by 
    \begin{enumerate}
        \item $\Inv(\Phi)$ the set of relations on $A$ that are preserved by all the elements of $\Phi$; 
        \item $\Pol(\Gamma)$ the set of operations on $A$ that preserve all the elements of $\Gamma$. 
    \end{enumerate}
    Moreover, if $\mathbf{A}$ is an algebra and $\EuScript{A}$ a relational structure on the same set $A$: 
    \begin{enumerate}
        \item $\Inv(\mathbf{A})$ is the set of relations that are preserved by all the $f^{\mathbf{A}}$;
        \item $\Pol(\EuScript{A})$ is the set of operations that preserve all the $r^{\EuScript{A}}$. 
    \end{enumerate} 
\end{definition}

%Let $F$ be a set of function symbols. 
%Let $\mathbf{A}=(A, \Phi)$ be an algebra over $F$. 
Let $\mathbf{A}$ be an algebra. 
Let $\EuScript{A}:=(A, \Inv(\mathbf{A}))$. 
By $\CSP(\mathbf{A})$ we mean the decision problem $\CSP(\EuScript{A})$. 
%decide uniformly, that is by a unique algorithm, every $\CSP((A,P))$ such that $P \subseteq \Inv(\Phi)$. 

\begin{definition}
    Let $F$ be a set of function symbols and $\mathbf{A}$ be an algebra over $F$. 
    We denote by $\Clo(\mathbf{A})$ the smallest set containing 
    \begin{equation*}
        \{f^\mathbf{A}: f \in F\} \quad \text{ and } \quad \{\pi^n_i: A^n \to A, 1 \le i \le n, n \in \omega\}
    \end{equation*}
and closed under composition. 
The elements of $\Clo(\mathbf{A})$ are called \textbf{term} operations. 
We say that two algebras $\mathbf{A}$ and $\mathbf{B}$ on the same carrier are \textbf{term equivalent} if $\Clo(\mathbf{A})=\Clo(\mathbf{B})$. 
\end{definition}

Recall that a variety is a class of algebras closed under homomorphic images, subalgebras and products. 
If $\mathsf{V}$ is a variety, we define $\Clo(\mathsf{V})$ to be $\Clo(\mathbf{F}_\mathsf{V}(\omega))$, where $\mathbf{F}_\mathsf{V}(\omega)$ is the $\mathsf{V}$-free algebra generated by a countable number of generators.
When $\mathsf{V}$ is $\mathsf{Set}$, the variety of sets, $\Clo(\mathsf{Set})$ is the clone of projections $\{\pi^n_i: 1 \le i \le n, n \in \omega\}$, that we denote by $\mathbf{N}$. 

Goal: prove 

\begin{theorem}
    Let $\mathbf{A}$ be a finite 
    idempotent 
    algebra. 
    Then the following are equivalent: 
    \begin{enumerate}
        \item $\CSP(\mathbf{A})$ is polynomial-time decidable; 
        \item $\Clo(\mathbf{A})$ contains a weak near-unanimity operation; 
        \item for every $\mathbf{B} \in HS(\mathbf{A})$, $\Clo(\mathbf{B}) \neq \mathbf{N}$. 
    \end{enumerate}
    Otherwise, $\CSP(\mathbf{A})$ is $\NP$-complete. 
\end{theorem}

Observe that 

\section{Kinds of Operations}

\begin{definition}
    An operation $\phi: A^n \to A$ is called 
    \begin{enumerate}
        \item \textbf{essentially unary} if there is an index $i$ and a non-constant function $\psi: A \to A$ such that 
        \begin{equation*}
        \phi(a_1, \ldots, a_n) = \psi(a_i)
        \end{equation*}
        for all $a_1, \ldots, a_n \in A$. 
        \item \textbf{idempotent} if $\phi(a, \ldots, a)=a$ for all $a \in A$. 
        \item a \textbf{near unanimity} operation if for all $a,b \in A$ 
        \begin{equation*}
            \phi(b, a, \ldots, a) = \phi(a, b, \ldots, a) = \cdots = \phi(a, \ldots, a, b)= a 
        \end{equation*}
    \end{enumerate}
\end{definition}

\begin{example}
    \label{dual}
    A ternary near unanimity operation is called a \textbf{majority} operation. 
    For instance the ternary function defined as 
    \begin{equation*}
        \delta(a,b,c) = 
        \begin{cases}
            b & \text{ if } b = c \\
            a & \text{ otherwise}
        \end{cases}
    \end{equation*}
    is a majority operation called \textbf{dual discriminator}. 
\end{example}

\section{Relational Clones} 
\begin{definition}
    Let $R$ be a set of relation symbols and $\EuScript{A}$ be a relational structure over $R$.
    We denote by $\Clo(\EuScript{A})$ the smallest set containing 
    \begin{equation*}
        \{r^\EuScript{A} : r \in R\} \quad \text{ and } \quad \{\Delta^{(n)} : n \in \omega\}
    \end{equation*}
    and closed under 
    \begin{enumerate}
        \item \textbf{permutation}: if $\rho \in \Clo(\EuScript{A})$, then also
        \begin{equation*}
            \{(a_{\sigma(1)}, \ldots, a_{\sigma(n)}) : \sigma \in S_n, (a_1, \ldots, a_n) \in \rho\} \in \Clo{\EuScript{A}}
        \end{equation*} 
        \item \textbf{extension}: if $\rho \in \Clo(\EuScript{A})$, then also
        \begin{equation*}
            \{(a_{1}, \ldots, a_{n}, a_{n+1}) :  (a_1, \ldots, a_n) \in \rho, a_{n+1} \in A\} \in \Clo{\EuScript{A}}
        \end{equation*} 
        \item \textbf{truncation}: if $\rho \in \Clo(\EuScript{A})$, then also
        \begin{equation*}
            \{(a_1, \ldots, a_{n-1}): (a_1, \ldots, a_{n-1}, a_n) \in \rho, \text{ for some } a_n \in A \} \in \Clo{\EuScript{A}}
        \end{equation*}
        \item intersection. 
    \end{enumerate}
\end{definition}

\begin{remark}
    Observe that $\Clo(\EuScript{A})$ is given by all the relations $\rho$ of $A$ definable by a first-order primitive positive formula
    (that is, involving only conjunctions and existential quantifications). 
    Recall that $\rho \subseteq A^n$ is definable if there is a formula $\phi(x_1, \ldots, x_n)$ such that 
    \begin{equation*}
        \EuScript{A} \models \phi(a_1, \ldots, a_n) \iff (a_1, \ldots, a_n) \in \rho
    \end{equation*}
\end{remark}

\begin{theorem}[\cite{jeavons}]
    %Let $G$ be a set and $D$ be a finite set of relation symbols. 
    For any pair of relational structures $\EuScript{A}=(A,\Gamma)$ 
    %over $G$ 
    and $\EuScript{B}=(A, H)$ 
    %over $D$ 
    such that $H$ is {finite} and $H \subseteq \Clo(\EuScript{A})$, $\CSP(\EuScript{B})$ is polynomial-time reducible to $\CSP(\EuScript{A})$.
    \begin{proof}
        Let $\Sigma$ be a set of atomic formulas over $H$.  
        Let $\eta(x_1, \ldots, x_n) \in \Sigma$. 
        Then there is $\phi(x_1, \ldots, x_n)$ of the form 
        \begin{equation*}
            \E y_1, \ldots, y_m \,( \gamma_1(z^{1}_{1}, \ldots, z^{1}_{n_1}) \land \cdots \land \gamma_k(z^{k}_{1}, \ldots, z^{k}_{n_k}))
        \end{equation*} 
        (where $\gamma_1, \ldots, \gamma_k \in \Gamma$ and $z^{i}_{j} \in \{x_1, \ldots, x_n, y_1, \ldots, y_m\}$)
        such that for $a_1, \ldots, a_n \in A$ 
        \begin{equation}
            \label{equiv}
            \EuScript{B} \models \eta(a_1, \ldots, a_n)  
            %\iff (a_1, \ldots, a_n) \in \delta 
            \iff \EuScript{A} \models \phi(a_1, \ldots, a_n)
        \end{equation}
        We can assume (up to renaming of variables) that $y_1, \ldots, y_m$ do not appear in any formula of $\Sigma$.  

        Now, for each $\eta(x_1, \ldots, x_n) \in \Sigma$ perform the following steps: 
        \begin{itemize}
            \item add $\{\gamma_1(z^{1}_{1}, \ldots, z^{1}_{n_1}), \ldots, \gamma_k(z^{k}_{1}, \ldots, z^{k}_{n_k})\}$ to $\Sigma$; 
            \item remove $\eta(x_1, \ldots, x_n)$ from $\Sigma$. 
        \end{itemize}
        At the the end we obtain a set of equations $T$ over $G$. 
        {This is a polynomial-time reduction. 
        %(It's reasonable but for me this kind of stuff is like a leap of faith).
        } 
        By \eqref{equiv} it is clear that we can find an assignment $X \to A$ such that $\EuScript{B} \models \Sigma$ iff we can find an assignment such that $\EuScript{A} \models T$.
    \end{proof} 
\end{theorem}

\begin{remark}
    We observe that in the above result it is not necessary that $A$ is a finite set. 
\end{remark}

\begin{corollary}
    Let $\EuScript{A}$ be a relational structure and $\EuScript{B}=(A, \Clo(\EuScript{A}))$. 
    Then 
    \begin{enumerate}
        \item $\CSP(\EuScript{A})$ is polynomial-time decidable iff $\CSP(\EuScript{B})$ is. 
        \item $\CSP(\EuScript{A})$ is $\NP$-complete iff $\CSP(\EuScript{B})$ is. 
    \end{enumerate}
    \begin{comment}
    \begin{proof}
        We prove that $\CSP(\EuScript{A})$ is polynomial-time equivalent to $\CSP(\EuScript{B})$. 
        Let $\EuScript{A}=(A,P)$. 
        Let $\Sigma$ be a finite set of atomic formulas over $P$. 

    \end{proof}
\end{comment}
\end{corollary}

\section{Surjective and Idempotent Algebras}

\begin{definition}
    An algebra $\mathbf{A}$ is \textbf{surjective} if all the element of $\Clo(\mathbf{A})$ are surjective. 
\end{definition}

\begin{remark}
    \label{surj_group}
    Let $\mathbf{A}$ be a finite algebra over $F$. 
    Every element of $\Clo(\mathbf{A})$ is surjective iff every element of $\Clo_1(\mathbf{A})$ is. 
    In this case $\Clo_1(\mathbf{A})$ is a group.
    It is enough to show that for every $f \in F$, there is $\phi \in \Clo_1(\mathbf{A})$ such that $f^\mathbf{A} \phi = 1_\mathbf{A}$. 
    Let $m:=|A|$. 
    Then $(f^{\mathbf{A}})^{m!}=1_\mathbf{A}$.
    Let $n_f$ be the least $n$ such that $(f^{\mathbf{A}})^{n_f}=1_\mathbf{A}$. 
    Let $\phi:=(f^{\mathbf{A}})^{n_f-1}$. 
\end{remark}

Let $\mathbf{A}$ be a finite algebra and let $B$ be a subset of $A$. 
Let 
\begin{equation*}
    \Clo(\mathbf{A})|B:=\{\phi \in \Clo(\mathbf{A}): \phi[B^n] \subseteq B \}
\end{equation*} 
We denote by $\mathbf{A}|B$ the algebra $(B, \Clo(\mathbf{A})|B)$. 

\begin{lemma}[\cite{jeavons}]
    \label{image}
    Let $\EuScript{A}=(A, P)$ be a relational structure. 
    Let $\phi \in \Pol_1(\EuScript{A})$. 
    For $\rho \in P_n$ let 
    \begin{equation*}
        \phi(\rho):=\{(\phi(a_1), \ldots, \phi(a_n)):(a_1, \ldots, a_n) \in \rho\}
    \end{equation*}
    Let $\EuScript{B}:=(A, \phi(P))$ where $\phi(P):=\{\phi(\rho): \rho \in P\}$.
    Then 
    \begin{enumerate}
        \item $\CSP(\EuScript{A})$ is polynomial-time decidable iff $\CSP(\EuScript{B})$ is; 
        \item $\CSP(\EuScript{A})$ is $\NP$-complete iff $\CSP(\EuScript{B})$ is. 
    \end{enumerate}
    \begin{proof}
        We show that $\CSP(\EuScript{A})$ is polynomial-time equivalent to $\CSP(\EuScript{B})$. 
        Let $\Sigma$ be a finite set of atomic formulas over $P$. 
        Replace every occurrence of $\rho(x_1, \ldots, x_n)$ with $\phi(\rho)(x_1, \ldots, x_n)$, 
        obtaining a set of formulas $T$ over $\phi(P)$. 
        Given any assignment $(-)^{\EuScript{B}}: X \to A$ such that $\EuScript{B} \models T$, 
        define $(x)^{\EuScript{A}}:=(x)^{\EuScript{B}}$; then $\EuScript{A} \models \Sigma$.
        Indeed, given $\rho(x_1, \ldots, x_n) \in \Sigma$, since $\phi \in \Pol_1(\EuScript{A})$: 
        \begin{align*}
            \EuScript{B} \models \phi(\rho)(x_1, \ldots, x_n) & \implies (x_1^{\EuScript{B}}, \ldots, x_n^{\EuScript{B}}) \in \phi(\rho) \\
            & \implies (x_1^{\EuScript{B}}, \ldots, x_n^{\EuScript{B}}) \in \rho \\
            & \implies (x_1^{\EuScript{A}}, \ldots, x_n^{\EuScript{A}}) \in \rho \\
            & \implies \EuScript{A} \models \rho(x_1, \ldots, x_n)
        \end{align*} 
        Conversely, given any assignment $(-)^{\EuScript{A}}: X \to A$ such that $\EuScript{A} \models \Sigma$,
        defining $(x)^{\EuScript{B}}:=\phi(x^{\EuScript{A}})$ is enough to have $\EuScript{B} \models T$.  
        Indeed, given $\phi(\rho)(x_1, \ldots, x_n) \in T$, by definition of $\phi(P)$: 
        \begin{align*}
            \EuScript{A} \models \rho(x_1, \ldots, x_n) & \implies (x_1^{\EuScript{A}}, \ldots, x_n^{\EuScript{A}}) \in \rho \\
            & \implies (\phi(x_1^{\EuScript{A}}), \ldots, \phi(x_n^{\EuScript{A}})) \in \phi(\rho) \\
            & \implies  (x_1^{\EuScript{B}}, \ldots, x_n^{\EuScript{B}}) \in \phi(\rho) \\
            & \implies \EuScript{B} \models \phi(\rho)(x_1, \ldots, x_n) \qedhere
        \end{align*} 
    \end{proof}
\end{lemma}

\begin{theorem}
    Let $\mathbf{A}$ be a finite algebra. 
    Then there is $B \subseteq A$ such that $\mathbf{A}|B$ is surjective and 
    \begin{enumerate}
        \item $\CSP(\mathbf{A})$ is polynomial-time decidable iff $\CSP(\mathbf{A}|B)$ is; 
        \item $\CSP(\mathbf{A})$ is $\NP$-complete iff $\CSP(\mathbf{A}|B)$.
    \end{enumerate}
    \begin{proof}
        Assume that $\mathbf{A}$ is not surjective. 
        Then by Remark \ref{surj_group} there is $\psi \in \Clo_1(\mathbf{A})$ not surjective. 
        Let $\phi \in \Clo_1(\mathbf{A})$ such that $\phi$ is not surjective and $\phi[A]$ has minimal cardinality. 
        Define $B:=\phi[A]$. 
        We show that $\mathbf{A}|B$ is surjective. 
        Let $\psi \in \Clo_1(\mathbf{A}|B) = \Clo_1(\mathbf{A})|B $;
        if, towards a contradiction, $\psi[B] \subset B$, then $\psi \phi [A] \subset \phi[A] \subset A$ contradicting the minimality. 
        The last part of the statement follows immediately from Lemma \ref{image}. 
    \end{proof}
\end{theorem}

\begin{definition}
    Let $\mathbf{A}$ be an algebra. 
    Let $\id(A)$ be the set of idempotent operations on $A$. 
    We define $\Clo_{\id}(\mathbf{A}): = \Clo(\mathbf{A}) \cap \id(A)$. 
    We say that $\mathbf{A}$ is \textbf{idempotent} of all the elements of $\Clo(\mathbf{A})$ are. 
\end{definition}

\begin{remark}
    \label{idempotent}
    Let $\mathbf{A}$ be an algebra.  
    Observe that an operation $\phi \in \Clo(\mathbf{A})$ is idempotent iff it preserves the relations in $\Delta^{(1)}=\{\{a\} : a \in A\}$. 
    Hence $\Inv(\Clo_{\id}(\mathbf{A})) = \Clo(\EuScript{A})$ where $\EuScript{A}=(A, \Inv(\mathbf{A}) \cup \Delta^{(1)})$. 
\end{remark}

\begin{lemma}
    \label{clo1_group}
    Let $A=\{a_1, \ldots, a_k\}$ be a finite set. 
    Let $\mathbf{A}$ be an algebra over $F$. 
    %{such that $\Clo_1(\mathbf{A})$ is a group -- where do we need it?} 
    Then the relation 
    \begin{equation}
        \sigma:=\{(\psi(a_1), \ldots, \psi(a_k)) : \psi \in \Clo_1(\mathbf{A})\}
    \end{equation}
    belongs to $\Inv(\mathbf{A})$. 
    \begin{proof}
        We show that for every $f \in F_n$ and for every matrix $M$
        \begin{equation*}
            \begin{bmatrix}
              a^1_1 & \cdots & a^1_n \\
              \vdots & \ddots &  \vdots  \\
              a^k_1 & \cdots & a^k_n
              \end{bmatrix}
          \end{equation*}
          such that $(a_1^1, \ldots, a^k_1) \in \sigma, \ldots, (a_1^n, \ldots, a^k_n) \in \sigma$ we have 
          \begin{equation*}
            (f^{\mathbf{A}}(a_1^1, \ldots, a^1_n), \ldots,f^{\mathbf{A}}(a^k_1, \ldots, a^k_n)) \in \sigma
          \end{equation*} 
          %Since $(a_1^i, \ldots, a^k_i) \in \sigma$, there is $\psi_i \in \Clo_1(\mathbf{A})$ such that 
          By hypothesis we can write $M$ as 
          \begin{equation*}
            \begin{bmatrix}
              \psi_1(a_1) & \cdots & \psi_n(a_1) \\
              \vdots & \ddots &  \vdots  \\
              \psi_1(a_k) & \cdots & \psi_n(a_k)
              \end{bmatrix}
          \end{equation*}
          but then 
        \begin{multline*}
            (f^{\mathbf{A}}(a_1^1, \ldots, a^1_n), \ldots,f^{\mathbf{A}}(a^k_1, \ldots, a^k_n))\\
             = (f^{\mathbf{A}}(\psi_1(a_1), \ldots, \psi_n(a_1)), \ldots,f^{\mathbf{A}}(\psi_1(a_k), \ldots, \psi_n(a_k))) \\
             = (f^{\mathbf{A}}[\psi_1, \ldots, \psi_n](a_1), \ldots, f^{\mathbf{A}}[\psi_1, \ldots, \psi_n](a_k))
        \end{multline*}
        and we conclude since $f^{\mathbf{A}}[\psi_1, \ldots, \psi_n] \in \Clo_1(\mathbf{A})$. 
    \end{proof}
\end{lemma}

\begin{theorem}
    \label{thm_surj}
    %Let $F$ be a set of function symbols and 
    Let $\mathbf{A}=A$  be a finite surjective algebra.
    Let $\mathbf{B}:=(A, \Clo_{\id}(\mathbf{A}))$. 
    Then   
    \begin{enumerate}
        \item $\CSP(\mathbf{A})$ is polynomial-time decidable iff $\CSP(\mathbf{B})$ is. 
        \item $\CSP(\mathbf{A})$ is $\NP$-complete iff $\CSP(\mathbf{B})$ is. 
    \end{enumerate}
\begin{proof}
    Let $A=\{a_1, \ldots, a_k\}$ and let $\Gamma:=\Delta^{(1)}$. 
    %For every $\rho \in \Inv(\Phi)$, let $r$ be a relation symbol of the same arity and let $R:=\{r: \rho \in \Inv(\Phi)\}$. 
    %For every $\delta_i:=\{a_i\} \in \Delta$, let $d_i$ be a relation symbol of the same arity and let $G:= R \cup \{d_i: \delta_i \in \Delta\}$. 
    Let $\EuScript{A}:=(A, \Inv(\mathbf{A}))$ and $\EuScript{B}:=(A, \Inv(\mathbf{A}) \cup \Gamma )$. 

    By definition and by Remark \ref{idempotent} $\CSP(\mathbf{B})$ is polynomial-time equivalent to $\CSP(\mathbf{A})$ iff 
    $\CSP(\EuScript{B})$ is polynomial-time equivalent to $\CSP(\EuScript{A})$.

    That $\CSP(\EuScript{A})$ is polynomial-time reducible to $\CSP(\EuScript{B})$ is obvious.
    Let $\Sigma$ be a finite set of atomic formulas over $\Inv(\mathbf{A}) \cup \Gamma$ and let $\{x_1, \ldots, x_k\}$ be variables that do not appear in $\Sigma$. 
    By Remark \ref{surj_group}, since $\mathbf{A}$ is surjective, $\Clo_1(\mathbf{A})$ forms a group. 
    Moreover, the relation $\sigma$ of Lemma \ref{clo1_group} belongs to $\Inv(\mathbf{A})$. 
    Now, perform the following steps: 
        \begin{itemize}
            \item replace every formula $\gamma_i(x)$ with $x \equiv x_i$; 
            \item add the formula $\sigma(x_1, \ldots, x_k)$. 
        \end{itemize}
    At the the end we obtain a set of equations $T$ over $\Inv(\mathbf{A})$. 
    {This is a polynomial-time reduction.}
    We finally show that we can find an assignment such that $\EuScript{A} \models T$ iff we can find an assignment such that $\EuScript{B} \models \Sigma$.
    Let $(-)^{\EuScript{B}}: X \to A$ be an assignment such that $\EuScript{B} \models \Sigma$. 
    Consider the assignment 
    \begin{equation*}
        x^{\EuScript{A}} = 
        \begin{cases}
            x^{\EuScript{B}} & \text{ if } x \neq x_i \\
            a_i & \text{ if } x = x_i 
        \end{cases}
    \end{equation*}
    Then $(-)^\EuScript{A}$ is such that $\EuScript{A} \models T$.  
    Conversely, assume that there is an assignment $(-)^{\EuScript{A}}$ such that $\EuScript{A} \models T$. 
    By definition of $\sigma$, there is $\psi \in \Clo_1(\mathbf{A})$ such that $x_i^{\EuScript{A}} = \psi(a_i)$ for all $i$. 
    Consider $(-)'^{\EuScript{A}}:=\psi^{-1}\circ(-)^{\EuScript{A}}$. 
    Define $(x)^{\EuScript{B}}:=(x)'^{\EuScript{A}}$.
    Let $\rho(y_1, \ldots, y_n), \gamma_i(x) \in \Sigma$. 
    Every relation in $\Inv(\mathbf{A})$ is invariant under $\psi^{-1}$, hence  
    \begin{alignat*}{2}
        \EuScript{A} \models \rho(y_1, \ldots, y_n) & \implies (y_1^{\EuScript{A}}, \ldots, y_n^{\EuScript{A}}) \in \rho  & \EuScript{A} \models x \equiv x_i &  \implies x^{\EuScript{A}} = x_i^{\EuScript{A}}  \\
        & \implies (\psi^{-1}(y_1^{\EuScript{A}}), \ldots, \psi^{-1}(y_n^{\EuScript{A}})) \in \rho  &   & \implies x^{\EuScript{A}}=\psi(a_i) \\
        & \implies (y_1^{\EuScript{B}}, \ldots, y_n^{\EuScript{B}}) \in \rho  &   & \implies \psi^{-1}(x^{\EuScript{A}}) = a_i   \\
        & \implies \EuScript{B} \models \rho(y_1, \ldots, y_n)  &    & \implies x^{\EuScript{B}}= a_i  
    \end{alignat*}
    Thus $\EuScript{B} \models \Sigma$. 
\end{proof}
\end{theorem}

\begin{remark}
    By the above result, given a finite algebra $\mathbf{A}$, to the purpose of the study of $\CSP(\mathbf{A})$, we can assume without loss of generality that $\mathbf{A}$ is idempotent. 
\end{remark}

\section{Subalgebras and Images}
\begin{theorem}
    \label{sub}
    Let $\mathbf{A}$ be a finite algebra. 
    \begin{enumerate}
        \item if $\CSP(\mathbf{A})$ is polynomial-time decidable, so is $\CSP(\mathbf{B})$ for every $\mathbf{B} \le \mathbf{A}$; 
        \item if there is $\mathbf{B} \le \mathbf{A}$ such that $\CSP(\mathbf{B})$ is $\NP$-complete, so is $\CSP(\mathbf{A})$. 
    \end{enumerate}
    \begin{proof}
    Let $\EuScript{A}=(A, \Inv(\mathbf{A}))$ and $\EuScript{B}=(A, \Inv(\mathbf{B}))$. 
    By definition $\CSP(\mathbf{B})$ is polynomial-time reducible to $\CSP(\mathbf{A})$ iff 
    $\CSP(\EuScript{B})$ is polynomial-time reducible to $\CSP(\EuScript{A})$.
    But that $\CSP(\EuScript{B})$ is polynomial-time reducible to $\CSP(\EuScript{A})$ is obvious since $\Inv(\mathbf{B}) \subseteq \Inv(\mathbf{A})$.
    \end{proof}
\end{theorem}

\begin{theorem}
    \label{homo}
    Let $\mathbf{A}, \mathbf{B}$ be two finite algebras of the same type.  
    \begin{enumerate}
        \item if $\CSP(\mathbf{A})$ is polynomial-time decidable, so is $\CSP(\mathbf{B})$ for every surjective homomorphism $\alpha: \mathbf{A} \to \mathbf{B}$; 
        \item if there is a surjective homomorphism $\alpha: \mathbf{A} \to \mathbf{B}$ such that $\CSP(\mathbf{B})$ is $\NP$-complete, so is $\CSP(\mathbf{A})$. 
    \end{enumerate}
    \begin{proof}
        %Let $\mathbf{A}=(A, \Phi_A)$, $\mathbf{B}=(B, \Phi_B)$. 
        Let $\EuScript{B}=(B, \Inv(\mathbf{B}))$.
        We show that there is $\Gamma \subseteq \Inv(\mathbf{B})$ such that, definining $\EuScript{A}=(A, \Gamma)$, $\CSP(\EuScript{B})$ is polynomial-time reducible to $\CSP(\EuScript{A})$. 
        For every $\rho \in \Inv(\mathbf{B})_n$ let 
        \begin{equation*}
            \alpha^{-1}(\rho):=\{(a_1, \ldots, a_n) \in A^n : (\alpha(a_1), \ldots, \alpha(a_n)) \in \rho\}
        \end{equation*}
        Clearly, $\alpha^{-1}(\rho) \in \Inv(\mathbf{A})_n$
        and therefore, letting $\Gamma:=\{\alpha^{-1}(\rho) : \rho \in \Gamma\}$, $\Gamma \subseteq \Inv(\mathbf{A})$. 
        Let $\Sigma$ be a set of atomic formulas over $\Inv(\mathbf{B})$. 
        Replace every formula $\rho(x_1, \ldots, x_n)$ with $\alpha^{-1}(\rho)(x_1, \ldots, x_n)$.       
        We obtain a set of equations $T$ over $\Gamma$ by a polynomial-time reduction. 
        Let $\beta$ be a section of $\alpha$.  
        \begin{equation*}
            \begin{tikzcd}
                %[column sep=small]
                & X \arrow[dl, "(-)^{\EuScript{A}}", swap] \arrow[dr, "(-)^{\EuScript{B}}"] & \\
                A \arrow[rr, bend right=15, "\alpha", swap] & & B \arrow[ll, bend right=15, dotted, "\beta"]
            \end{tikzcd}
        \end{equation*}
        Referring to the assignments in the picture, each defined in terms of the other so that the diagram commute, 
        it is clear that $\EuScript{A} \models T$ iff $\EuScript{B} \models \Sigma$. 
    \end{proof}
\end{theorem}


\begin{lemma} [\cite{jeavons}]
    \label{ess-unary}
    Let $\EuScript{A}$ be a relational structure. 
    If $\Pol(\EuScript{A})$ contains essentially unary operations only, $\CSP(\EuScript{A})$ is $\NP$-complete. 
\end{lemma}

\begin{corollary}
    Let $\mathbf{A}$ be a finite 
    algebra. 
    If there is $\mathbf{B} \in HS(\mathbf{A})$ such that $\Clo(\mathbf{B}) = \mathbf{N}$, then 
    $\CSP(\mathbf{A})$ is $\NP$-complete. 
    \begin{proof}
        If $\mathbf{B}$ is such that $\Clo(\mathbf{B}) = \mathbf{N}$, then $\CSP(\mathbf{B})$ is $\NP$-complete by Lemma \ref{ess-unary}. 
        We conclude using the second cluases of Theorems \ref{sub} and \ref{homo}. 
   \end{proof}
\end{corollary}

\begin{comment}
\section{Simple and Striclty Simple Algebras}
\begin{definition}
    An algebra $\mathbf{A}$ is called \textbf{simple} if the lattice of congruences on $\mathbf{A}$ is $\{\Delta, \nabla \}$. 
    An algebra $\mathbf{A}$ is \textbf{strictly simple} if it is simple and has no nontrivial subalgebras. 
\end{definition}

\begin{remark}
    Since the lattice of congruence of $\mathbf{A}$ is in bijection with the set of surjective homomorphisms from $\mathbf{A}$, 
    $\mathbf{A}$ is simple if whenever there is a surjective homomorphism $\mathbf{A} \to \mathbf{B}$ such that $|B| < |A|$, then $\mathbf{B}$ is the trivial algebra. 
\end{remark}

\subsection{Group actions}
Let $\mathbf{G}$ be a group acting on a set $A$.  

There is a well-defined function
\begin{equation*}
    G \times A \to A \times A, (g,a) \mapsto (a, g \cdot a)
\end{equation*} 

\begin{definition}
    The action is called 
    \begin{enumerate}
        \item \textbf{transitive} if the above map is surjective; 
        \item \textbf{free} if the above map is injective; 
        \item \textbf{regular} if the above map is bijective. 
    \end{enumerate}
    We say that the group $\mathbf{G}$ is \textbf{regular} if the action is. 
\end{definition}

\begin{definition}
    We say that $\mathbf{G}$ is (or, more accurately, that the action is)
   %\begin{enumerate}
        %\item \textbf{regular} if the action is transitive and free; 
        %\item 
        \textbf{primitive} if $\{A\}$ and $\{\{a\} : a \in A\}$ are the only partitions invariant under $\mathbf{G}$. 
    %\end{enumerate}
\end{definition}

\begin{remark}
    Each $g \in G$ induces 
    \begin{enumerate}
        \item a relation $\rho_g:=\{(a, g\cdot a) : a \in A\} \subseteq A^2$. 
        Let $P_{\mathbf{G}}:=\{\rho_g : g \in G\}$. 
        \item an operation $\phi_g: A \to A$ given by $\phi_g(a)=g \cdot a$. 
        Let $\Phi_\mathbf{G}:=\{\phi_g: g \in G\}$. 
    \end{enumerate}
    Thus a $\mathbf{G}$-set can be seen as an algebra $(A, \Phi_\mathbf{G})$. 
    The action is primitive iff this algebra is simple.
\end{remark}

%Let $\mathcal{R}(\mathbf{G}):=\Pol(P_{\mathbf{G}})$. 

\subsection{The classification theorem}

\begin{comment}
\begin{theorem}
    [\cite{classification}]
    Let $\mathbf{A}$ be a finite strictly simple surjective algebra. 
    If $\mathbf{A}$ has trivial subalgebras, $\mathbf{A}$ is term-equivalent to one of the following: 
    \begin{enumerate}
        \item \label{one} $(A,\Pol(P_{\mathbf{G}}))$ for some regular group $\mathbf{G}$;
        \item \label{three} $(A, \Phi_\mathbf{G})$ for some primitive group $\mathbf{G}$; 
    \end{enumerate}
    or 
    \begin{enumerate}
        \setcounter{enumi}{2}
        \item there is a finite field $k$ and a $k$-vector space $\mathbf{E}$ with carrier $A$ such that letting $\mathbf{E}'$ be the $\End_k(\mathbf{E})$-module with natural action 
    \begin{equation*}
        \End_k(\mathbf{E}) \times A \to A, (\eta, a) \to \eta(a)  
    \end{equation*}
    $\mathbf{A}$ is term-equivalent to $(A, \Clo_{\id}(\mathbf{E}') \cup T(\mathbf{E}))$. 
    \end{enumerate}
    If $\mathbf{A}$ has nontrivial subalgebras, $\mathbf{A}$ is term-equivalent to one of the following: 
    \begin{enumerate}
        \item 
    \end{enumerate}
\end{theorem}

\begin{lemma}
    Let $\EuScript{A}$ be a relational structure. 
    If $\Pol(\EuScript{A})$ contains a majority operation, $\CSP(\EuScript{\EuScript{A}})$ is polynomial-time decidable. 
\end{lemma}

\begin{theorem}
    %\label{ess-unary}
    Let $\mathbf{A}$ be a finite strictly simple surjective algebra.
    If all the elements of $\Clo(\mathbf{A})$ are essentially unary, then $\CSP(\mathbf{A})$ is $\NP$-complete. 
    Otherwise, $\CSP(\mathbf{A})$ is polynomial-time decidable. 
    \begin{proof}
        %If $\mathbf{A}$ is of kind \eqref{one}, we prove that the dual discriminator operation $\delta$ of Example \ref{dual} belongs to $\Clo(\mathbf{A})$. 
        %But $\delta \in \Clo(\mathbf{A})$ if, by definition of $\mathcal{R}(\mathbf{G})$, 

    \end{proof}
\end{theorem}
\end{comment}

\section{Omitting Types and Complexity}
Refer to the Appendices. 

\begin{comment}
Let $\mathbf{A}$ be a finite algebra and let $U$ be a subset of $A$. 
In this section, let 
\begin{equation*}
    \Pol(\mathbf{A})|U:=\{\psi \in \Pol(\mathbf{A}): \psi[U^n] \subseteq U \}
\end{equation*} 
We denote by $\mathbf{A}|U$ the algebra $(A, \Pol(\mathbf{A})|U)$. 

\begin{lemma}
    \label{pol}
    Let $\mathbf{A}$ be a finite idempotent algebra. 
    If $e \in \Pol_1(\mathbf{A})$ and $U=e(A)$, then 
    \begin{enumerate}
        \item $\CSP(\mathbf{A})$ is polynomial-time decidable iff $\CSP(\mathbf{A}|U)$ is. 
        \item $\CSP(\mathbf{A})$ is $\NP$-complete iff $\CSP(\mathbf{A}|U)$ is. 
    \end{enumerate}
    \begin{proof}
        We show that 
    \end{proof}
\end{lemma}
\end{comment}

\begin{theorem}
    Let $\mathbf{A}$ be a finite idempotent algebra. 
    If $\CSP(\mathbf{A})$ is decidable in polynomial-time, then $\mathbf{1} \notin \typ\{\mathbf{A}\}$.  
    %If $\mathbf{1} \in \typ\{\mathbf{A}\}$, then $\CSP(\mathbf{A})$ is $\NP$-complete. 
    \begin{proof}
        Assume for the sake of condtradiction that $\mathbf{1} \in \typ\{\mathbf{A}\}$. 
        We show that $\CSP(\mathbf{A})$ is $\NP$-complete. 
        If $\mathbf{1} \in \typ\{\mathbf{A}\}$, then there are $\alpha, \beta \in \Con(\mathbf{A})$ such that $\typ(\alpha, \beta) = \mathbf{1}$.
        By Theorem \ref{homo}, up to replacing $\mathbf{A}$ by $\mathbf{A}/\alpha$, we can assume without loss of generality that $\alpha = \Delta_A$. 

        Therefore $\Clo(\mathbf{C}|V)$ contains essentially unary operations only. 
        By Lemma \ref{ess-unary} $\CSP(\mathbf{C}|V)$ is $\NP$-complete. 
        By Lemma \ref{image} $\CSP(\mathbf{B}|U)$ and $\mathbf{B}$ are $\NP$-complete. 
        Since $\Clo(\mathbf{A}) \subseteq \Clo(\mathbf{B})$, $\mathbf{A}$ is $\NP$-complete. 
    \end{proof}
\end{theorem}

\appendix
\section{Classification of Finite Minimal Algebras}

\begin{definition}
    Let $F$ be a set of function symbols and $\mathbf{A}$ be an algebra over $F$. 
    We denote by $\Pol(\mathbf{A})$ the smallest set containing 
    \begin{enumerate}
        \item $\{f^\mathbf{A}: f \in F\}$; 
        \item $\{\pi^n_i: A^n \to A, 1 \le i \le n, n \in \omega\}$; 
        \item the constant $0$-ary operations
    \end{enumerate}
and closed under composition. 
The elements of $\Pol(\mathbf{A})$ are called \textbf{polynomial} operations. 
We say that two algebras $\mathbf{A}$ and $\mathbf{B}$ on the same carrier are \textbf{polynomial equivalent} if $\Pol(\mathbf{A})=\Pol(\mathbf{B})$. 
\end{definition}


\begin{example}
    If $\phi \in \Clo_{m+n}(\mathbf{A})$ and $(a_1, \ldots, a_m) \in A^m$, then 
    \begin{equation*}
        \psi: A^n \to A \quad (b_1, \ldots, b_n) \mapsto \phi(a_1, \ldots, a_m,b_1, \ldots, b_n)
    \end{equation*}
is a polynomial operation.  
\end{example}

\begin{definition}
    [*]
    A nontrivial finite algebra $\mathbf{A}$ is \textbf{minimal} iff every noncostant element of $\Pol_1(\mathbf{A})$ is bijective.
\end{definition}

The goal is to classify, up to polynomial equivalence, all the finite minimal algebras. 

\begin{example}
    The following are examples of minimal algebras. 
    \begin{enumerate}
        \item any algebra with carrier $2$; 
        \item a nontrivial finite vector space $\mathbf{A}$ over a finite field $\mathbf{k}$: every $\pi \in \Pol_1(\mathbf{A})$ is of the form $\pi(v)=av+b$ for some $a \in k$, $b \in A$; 
        \item a group of permutations acting on a finite set\footnote{Let $\mathbf{G}$ be a group acting on a set $A$.  
        Each $g \in G$ induces an operation $\phi_g: A \to A$ given by $\phi_g(a)=g \cdot a$. 
        Let $\Phi_\mathbf{G}:=\{\phi_g: g \in G\}$. 
        A $\mathbf{G}$-set can be seen as an algebra $(A, \Phi_\mathbf{G})$.}.
    \end{enumerate}
\end{example}

We shall prove that, up to polynomial equivalence, there are no other finite minimal algebras. 

\begin{lemma}
    Let $\mathbf{A}$ be a minimal algebra.
    If every element of $\Pol(\mathbf{A})$ is essentially unary, then $\mathbf{A}$ is polynomial equivalent to $(A, \Phi_{\mathbf{G}})$ where $\mathbf{G}$ is a finite group acting on $A$. 
    \begin{proof}
        Since $\mathbf{A}$ is minimal, $\Pol_1(\mathbf{A}) \cap \Sym(A)$ is a subgroup of $\Sym(A)$. 
        Let $\mathbf{G}:=\Pol_1(\mathbf{A}) \cap \Sym(A)$. 
        If $\psi \in \Pol(\mathbf{A})$, either $\psi$ is constant or $\psi$ is essentially unary, hence $(A, \Phi_{\mathbf{G}})$ is polynomially equivalent to $\mathbf{A}$. 
    \end{proof}
\end{lemma}

\begin{theorem}
    [\cite{classification-minimal}]
    Let $\mathbf{A}$ be a minimal algebra with $|A| > 2$.  
    If $\Pol(\mathbf{A})$ contains an operation which is not essentially unary, then $\mathbf{A}$ is polynomially equivalent to a $\mathbf{k}$-vector space for a finite field $\mathbf{k}$. 
\end{theorem}

\begin{theorem}
    \label{classification-two}
    Every algebra $\mathbf{A}$ with carrier $2$ is polynomially equivalent to one of the following: 
    \begin{enumerate}
        \item $\mathbf{E}_0= (2, \varnothing)$;
        \item $\mathbf{E}_1 = (2, \lnot)$; 
        \item $\mathbf{E}_3 = (2, \land, \lor, \lnot)$; 
        \item $\mathbf{E}_4 = (2, \land, \lor)$; 
        \item $\mathbf{E}_5 = (2, \lor)$; 
        \item $\mathbf{E}_6 = (2, \land)$.
    \end{enumerate}
    Each of them is not polynomially equivalent to the other\footnote{A classical theorem by Post states that the set of clones of operations on $2$ is countable infinite. 
    By Theorem \ref{classification-two} among these there are exactly seven distinct clones containing the constant operations.
    However it has been proven that the set of clones on $3$ containing the constant operations is uncountable.}.
\end{theorem}

\begin{remark}
    Up to isomorphism, $\mathbf{E}_5 (\simeq \mathbf{E}_6)$ is the only semilattice with two elements, while $\mathbf{E}_3$ and $\mathbf{E}_4$ are the only Boolean algebra and lattice, respectively, with two elements. 
\end{remark}

\begin{definition}
    Let $\mathbf{A}$ be a minimal algebra. 
    We say that $\mathbf{A}$ is of 
    \begin{enumerate}
        \item \textbf{type 1} (or \textbf{unary}) if $\mathbf{A}$ is polynomially equivalent to $(A, \Phi_\mathbf{G})$ for some $\mathbf{G} \le \Sym(A)$; 
        \item \textbf{type 2} (or \textbf{affine}) if $\mathbf{A}$ is polynomially equivalent to a vector space over a finite field $\mathbf{k}$; 
        \item \textbf{type 3} (or \textbf{Boolean}) if $\mathbf{A}$ is polynomially equivalent to $\mathbf{E}_3$;
        \item \textbf{type 4} (or \textbf{lattice}) if $\mathbf{A}$ is polynomially equivalent to $\mathbf{E}_4$;
        \item \textbf{type 5} (or \textbf{semilattice}) if $\mathbf{A}$ is polynomially equivalent to $\mathbf{E}_5$. 
    \end{enumerate}
\end{definition}

\begin{definition}[*]
    Let $\mathbf{A}$ be a finite algebra and let $\delta < \theta \in \Con(\mathbf{A})$. 
    We say that $\mathbf{A}$ is $(\delta, \theta)$-\textbf{minimal} if for all $e \in \Pol_1(\mathbf{A})$ either $e$ is bijective or $e(\theta) \subseteq \delta$. 
\end{definition}

\begin{remark}
    Observe that $\mathbf{A}$ is minimal iff $\mathbf{A}$ is $(\Delta, \nabla)$-minimal. 
\end{remark}

\section{Omitting Types}

\begin{definition}
    Let $\mathsf{V}$ be a variety. 
    An algebra $\mathbf{A} \in \mathsf{V}$ is called
    \begin{enumerate}
        \item \textbf{free} if there is an isomorphism $\mathbf{A} \simeq \mathbf{F}_{\mathsf{V}}(\kappa)$ for some cardinal $\kappa$; 
        \item \textbf{finitely generated} if there is a surjective homomorphism $\mathbf{F}_{\mathsf{V}}(n) \to \mathbf{A}$ for some $n \in \omega$. 
    \end{enumerate} 
\end{definition}

\begin{definition}
    A variety $\mathsf{V}$ is called 
    \begin{enumerate}
    \item \textbf{locally finite} if all its finitely generated algebras are finite; 
    %\item \textbf{finitely presented} if $\mathsf{V}$ has a finite set of function symbols and $\mathsf{V}= \Alg(\Sigma)$ for a finite set of equations $\Sigma$; 
    \item \textbf{finitely generated} if $\mathsf{V}=V(\mathbf{A}_1, \ldots, \mathbf{A}_n)$ for $\mathbf{A}_1, \ldots, \mathbf{A}_n$ finite similar algebras. 
    \end{enumerate}
\end{definition}

\begin{remark}
    Observe that $V(\mathbf{A}_1, \ldots, \mathbf{A}_n)=V(\mathbf{A}_1 \times \cdots \times \mathbf{A}_n)$.  
\end{remark}

\begin{lemma}
    \label{loc_fin}
    Let $\mathsf{V}$ be a variety. 
    If $\mathsf{V}$ is finitely generated then it is locally finite. 
    \begin{proof}
        %Let $\mathsf{V}=HSP(\mathbf{A}_1, \ldots, \mathbf{A}_n)$ with $\mathbf{A}_i$ finite. 
        %Let $\mathbf{A} \in \mathsf{V}$ be finitely generated. 
        %Then there is a surjective homomorphism $\alpha: \mathbf{A}' \to \mathbf{A}$ for some 
        %\begin{equation*}
        %    \mathbf{A}' \le \mathbf{A}^{\kappa_1}_1 \times \cdots \times \mathbf{A}^{\kappa_n}_n
        %\end{equation*} 
        %We prove that $\mathbf{A}'$ is finite. 
        %Without loss of generality we can assume that $\mathbf{A}'$ is finitely generated: if not we can replace $\mathbf{A}'$ with $\mathbf{A}' /\ker(\alpha)$. 
        %Since $\mathbf{A}'$ is finitely generated, 
        %there is a finite number of homomorphisms $\mathbf{A}' \to \mathbf{A}_i$ for every $i$. 
        %Composing the embedding 
        %\begin{equation*}
        %    \mathbf{A}' \hookrightarrow  \mathbf{A}^{\kappa_1}_1 \times \cdots \times \mathbf{A}^{\kappa_n}_n
        %\end{equation*} with the projections, 
        %we see that $\mathbf{A}'$ embeds into $\mathbf{A}^{k_1}_1 \times \cdots \times \mathbf{A}^{k_n}_n$ for some \emph{finite} $k_1, \ldots, k_n$, hence it is finite. 
    
        By the previuos remark, we can assume that $\mathsf{V}=V(\mathbf{A})$ for some $\mathbf{A}$ finite. 
        Let $n < \omega$. 
        We prove that $\mathbf{F}_{\mathsf{V}}(n)$ is finite. 
        Consider the homomorphism
        \begin{equation*}
            \mathbf{F}_{\mathsf{V}}(n) \to \mathbf{A}^{A^n}, \quad t(x_1, \ldots, x_n) \mapsto t^\mathbf{A}
        \end{equation*}
        This homomorphism is injective: if $t^\mathbf{A} = s^\mathbf{A}$, then $\mathbf{A} \models t \equiv s$, i.e. $t=s$ in $\mathbf{F}_{\mathsf{V}}(n)$. 
        %between $\mathbf{F}_{\mathsf{V}}(n)$ and the subalgebra of $\mathbf{A}^{A^n}$ with underlying set $\Clo_n(\mathbf{A})$. 
        Thus $\mathbf{F}_{\mathsf{V}}(n)$ is finite. 
    \end{proof}
\end{lemma}

\begin{definition}
    Let $\mathsf{V}$ and $\mathsf{W}$ be two varieties. 
    We say that $\mathsf{V}$ is \textbf{interpretable} into $\mathsf{W}$ ($\mathsf{V} \le \mathsf{W}$) if there is a clone homomorphism $\Clo(\mathsf{V}) \to \Clo(\mathsf{W})$. 
\end{definition}

\begin{theorem}
    \label{omitting_typ1}
    Let $\mathsf{V}$ be a locally finite variety. 
    The following are equivalent: 
    \begin{enumerate}
        \item $\mathbf{1} \notin \typ\{\mathsf{V}\}$; 
        \item there is an idempotent variety $\mathsf{W}$ such that $\mathsf{W} \le \mathsf{V}$ and $\mathsf{W} \nleq \mathsf{Set}$. 
    \end{enumerate}
\end{theorem}

\begin{corollary}
    Let $\mathbf{A}$ be a finite idempotent algebra. 
    There is $\mathbf{B} \in HS(\mathbf{A})$ such that $\Clo(\mathbf{B})=\mathbf{N}$ iff $\mathbf{1} \in \typ\{HS(\mathbf{A})\}$. 
    \begin{proof}
        If $\mathbf{1} \in \typ\{HS(\mathbf{A})\}$, then $\mathbf{1} \in \typ\{HSP(\mathbf{A})\}$. 
        Since $\mathbf{A}$ is finite, then, by Lemma \ref{loc_fin} $HSP(\mathbf{A})$ is locally finite, 
        and therefore, by Theorem \ref{omitting_typ1}, for every idempotent variety $\mathsf{W}$, either $\mathsf{W} \nleq HSP(\mathbf{A})$ or $\mathsf{W} \le \mathsf{Set}$. 
        In particular, since $\mathbf{A}$ is idempotent, $HSP(\mathbf{A}) \le \mathsf{Set}$. 
        This means that there is a clone homomorphism $\Clo(\mathbf{A}) \to \mathbf{N}$. 
        Equivalently, every term operation of $\mathbf{A}$ is a projection. 
        Hence, $\Clo(\mathbf{A}) = \mathbf{N}$. 

        Conversely, let $\mathbf{B} \in HS(\mathbf{A})$ such that $\Clo(\mathbf{B})=\mathbf{N}$; 
        this means that $\mathbf{B}$ is term equivalent to a set. 
        Hence $\mathbf{B}$ is polynomial equivalent to a set on which the trivial group acts. 
        Then $\mathbf{1} \in \typ\{\mathbf{B}\}$, and therefore $\mathbf{1} \in \typ\{HS(\mathbf{A})\}$. 
    \end{proof}
\end{corollary}

\begin{comment}
\begin{corollary}
    Let $\mathbf{A}$ be a finite idempotent algebra. 
    Then $\mathbf{1} \in \typ\{HS(\mathbf{A})\}$ iff $HSP_{\fin}(\mathbf{A})$ contains a set. 
\end{corollary}
\end{comment}

\begin{theorem}
    [\cite{wnu}]
    Let $\mathbf{A}$ be a finite idempotent algebra. 
    Then $\Clo(\mathbf{A})$ contains a weak near-unanimity operation iff $\mathbf{1} \notin \typ\{HS(\mathbf{A})\}$. 
\end{theorem}

\begin{thebibliography}{9}
    \bibitem{bulatov-jeavons}
    Bulatov, A. A.,  Jeavons, P., Krokhin, A. A. (2005). Classifying the complexity of constraints using finite algebras. \emph{SIAM Journal of Computation} 34(3), 720-742.
    \bibitem{jeavons}
    Jeavons, P. (1998). On the algebraic structure of combinatorial problems, \emph{Theoretical Computer Science} 200, 185-204.
    \bibitem{wnu}
    Mar\'oti, M., McKenzie, R. (2008). Existence theorems for weakly symmetric operations. \emph{Algebra Universalis} 59, 463-489.
    \bibitem{classification-minimal} 
    P\'alfy, P. P. (1984). Unary polynomials in algebras I. \emph{Algebra Universalis} 18, 262-273.
    \bibitem{classification}
    Szendrei, Á. (1990). Simple surjective algebras having no proper subalgebras. \emph{Journal of the Australian Mathematical Society}, 48(3), 434-454. 
 \end{thebibliography}
\end{document}